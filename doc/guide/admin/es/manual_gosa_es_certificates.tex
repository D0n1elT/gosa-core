\chapter{Seguridad y Certificados}
\section{Introduction SSL}
\section{Creaci�n de certificados}
La seguridad es uno de los puntos mas importantes al configurar un servidor, necesitaremos un entorno seguro donde no permitir que los usuarios manipulen y accedan a codigo o programas.

Una formas de conseguir esto es usando encriptaci�n, con lo que buscamos que los usuarios y el servidor se comuniquen de forma que nadie mas pueda acceder a los datos. Esto se consigue con encriptaci�n.

La otra manera de asegurar el sistema es que si existe alg�n fallo en el sistema o en el c�digo, y un intruso intenta ejecutar codigo, este se vea incapacitado, ya que existen poderosas limitaciones, como no permitir que ejecute comandos, lea el codigo de otros script, no pueda modificar nada y tenga un usuario con muy limitados recursos.
\subsection{ Certificados SSL}

\noindent Existe amplia documentaci�n sobre encriptaci�n y concretamente sobre SSL, un sistema de encriptaci�n con clave publica y privada.\\
\\
\noindent Como el paquete openSSL ya lo tenemos instalado a partir de los pasos anteriores, debemos crear los certificados que usaremos en nuestro servidor web.\\
\\
\noindent Supongamos que guardamos el certificado en /etc/apache2/ssl/gosa.pem\\
\\
\begin{tabular}{|l|}\hline 
\#>FILE=/ect/apache2/ssl/gosa.pem\\
\#>export RANDFILE=/dev/random\\
\#>openssl req -new -x509 -nodes -out \$FILE -keyout /etc/apache2/ssl/apache.pem\\
\#>chmod 600 \$FILE\\
\#>ln -sf \$FILE /etc/apache2/ssl/`/usr/bin/openssl x509 -noout -hash < \$FILE`.0\\
\hline \end{tabular}
\vspace{0.5cm}
\\
\noindent Con esto hemos creado un certificado que nos permite el acceso SSL a nuestras p�ginas.\\
\\
\noindent Si lo que queremos es una configuraci�n que nos permita no solo que el tr�fico est� encriptado, sino que adem�s el cliente garantice que es un usuario v�lido, debemos provocar que el servidor pida una certificaci�n de cliente. \\
\\
\noindent En este caso seguiremos un procedimiento mas largo, primero la creaci�n de una certificaci�n de CA:\\
\\
\begin{tabular}{|l|}\hline 
\#>CAFILE=gosa.ca\\
\#>KEY=gosa.key\\
\#>REQFILE=gosa.req\\
\#>CERTFILE=gosa.cert\\
\#>DAYS=2048\\
\#>OUTDIR=.\\
\#>export RANDFILE=/dev/random\\
\#>openssl req -new -x509 -keyout \$KEY -out \$CAFILE -days \$DAYS\\
\hline \end{tabular}
\vspace{0.5cm}
\\
\noindent Despu�s de varias cuestiones tendremos una CA, ahora hacemos un requerimiento para un nuevo certificado:\\
\\
\begin{tabular}{|l|}\hline 
\#>DAYS=365\\
\#>openssl req -new -keyout \$REQFILE -out \$REQFILE -days\$DAYS\\
\hline \end{tabular}
\vspace{0.5cm}
\\
\noindent Creamos una configuraci�n para usar la CA con openssl y la guardamos en openssl.cnf:\\
\\
\begin{tabular}{|l|}\hline 
\verb|HOME = .|\\
\verb|RANDFILE = $ENV::HOME/.rnd|\\
\verb|[ ca ]|\\
\verb|default_ca  = CA_default|\\
\verb|[ CA_default ]|\\
\verb|dir = .|\\
\verb|database = index.txt|\\
\verb|serial = serial|\\
\verb|default_days = 365|\\
\verb|default_crl_days= 30|\\
\verb|default_md = md5|\\
\verb|preserve = no|\\
\verb|policy = policy_anything|\\
\verb|[ policy_anything ]|\\
\verb|countryName = optional|\\
\verb|stateOrProvinceName  = optional|\\
\verb|localityName = optional|\\
\verb|organizationName = optional|\\
\verb|organizationalUnitName  = optional|\\
\verb|commonName = supplied|\\
\verb|emailAddress = optional|\\
\hline \end{tabular}
\vspace{0.5cm}
\\
\noindent Firmamos el nuevo certificado:\\
\\
\begin{tabular}{|l|}\hline 
\#>\verb|touch index.txt|\\
\#>\verb|touch index.txt.attr|\\
\#>\verb|echo "01" >serial|\\
\#>\verb|openssl ca -config openssl.cnf -policy policy_anything \|\\\verb|-keyfile $KEY -cert $CAFILE -outdir . -out $CERFILE -infiles $REQFILE|\\
\hline \end{tabular}
\vspace{0.5cm}
\\
\noindent Y creamos un pkcs12 para configurar la certificaci�n en los clientes:\\
\\
\begin{tabular}{|l|}\hline 
\#>openssl pkcs12 -export -inkey \$KEY -in \$CERTFILE -out certificado\_cliente.pkcs12\\
\hline \end{tabular}
\vspace{0.5cm}
\\
\noindent Este certificado se puede instalar en el cliente, y en el servidor mediante la configuraci�n explicada en cada uno, esto nos dar� la seguridad de que su comunicaci�n ser� estrictamente confidencial.\\
